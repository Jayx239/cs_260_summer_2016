\documentclass{article}
	\usepackage{amsmath} %Needed f o r P r e t t y A l i g n s
	\usepackage{listings} % i n l i n e code
	\usepackage{algorithm2e} %P sued oc ode

	\begin{document}
		\title{CS 260 Homework 2}
		\date{7/7/2016}
		\author{Jason Gallagher}
		\maketitle

		\section{Problem 1.11}
$
\begin{array}{cc}
   g(1)=\{ & 
    \begin{array}{ccc}
      n^2 & even &  x\geq 0 \\
      n^3 &  odd & x\geq 1 \\
    \end{array}
\end{array}
$

$
\begin{array}{cc}
   g(2)=\{ & 
    \begin{array}{cc}
      n &  0 \leq x \geq 100 \\
      n^3 &  x\geq 100 \\
    \end{array}
\end{array}
$

$
\begin{array}{cc}
   g(3)=\{ & 
    \begin{array}{cc}
      n^2 &  x\geq 0 \\
      n^3 &  x\geq 1 \\
    \end{array}
\end{array}
$
\\
		\ Indicate for each distinct pair i and j whether gi(n) is O(gj(n)) and whether gi(n)is Ω(gj(n))\\
		
		
		\begin{lstlisting}
		
			g1	g2	g3
		O(g1)	t	t	f
		om(g1)	t	t	f
		
		O(g2)	t	t	f
		om(g2)	t	t	f
		
		O(g3)	f	f	t
		om(g3)	f	f	t
		\end{lstlisting}

		\section{Problem 1.13}

Show that the following statements are true. 
			\begin{enumerate}
				\item \begin{align*}17 <= 17n^0	\\for: n=any\end{align*}
				\item \begin{align*}n(n-1)/2 = (n^2-n)/2   \\(n^2-n)/2 <= 1/2n^2\\	 	n^2-n <= n^2\\when:  n>=0		\end{align*}
				\item \begin{align*}10n^2 < 10n^3\\	 n^2 <=n^3\\	for: n >= 10\end{align*}
				\item \begin{align*}
				1^k+2^k .... + n^k <=O(n^(k+1))\\
				1^k+2^k .... + n^k <= n^k+1\\
				1^k+2^k .... + n^k <= n*n^k\\
				(1/n)^k + (2/n)^k .... n^k-1 <= n^k\\
				\end{align*}
			\end{enumerate}



		\section{Problem 1.16}
		\begin{enumerate}
			\item \begin{equation*}\frac{3}2^n \end{equation*}
			\item \begin{equation*} \frac{1}3^n \end{equation*}
			\item \begin{equation*} \sqrt{n}log^2(n) \end{equation*}
			\item \begin{equation*} \frac{n}log(n) \end{equation*}
			\item \begin{equation*} log^2(n) \end{equation*}
			\item \begin{equation*} \sqrt{n} \end{equation*}
			\item \begin{equation*} log(n) \end{equation*}
			\item \begin{equation*} log(log(n)) \end{equation*}
			\item \begin{equation*} 17 \end{equation*}
		\end{enumerate}
		
		
		
		\section{Problem 1.18}

		\begin{enumerate}
		\item[A.] \begin{equation*}
		t(j) = \sum_{0}^{j/2}2^j
		\end{equation*}
		
		\item[B] \begin{equation*}
			t(j) = O(2^n)
		\end{equation*}
	\end{enumerate}
	
	
	
	
		\section{Problem 2.9}
	\	Write a procedure to interchange the elements at positions p and NEXT(p) in a singly linked list. The following procedure was intended to remove all occurrences of element x from list L. Explain why it doesn't always work and suggest a way to repair the procedure so it performs its intended task. 

	\begin{algorithm}
	\KwData{One elementtype and one list as input}
	\KwResult{List with element type removed}
	
	\begin{tabbing}
 proced\=ure delete(elementType x, list l) \+ \\
	  p = position \\
	  whil\=e p != END(L) do begin \+ \\
		if RE\=TRIEVE(p,L) = x then \+\\
			DELETE(p,L); \-\\
		  p = NEXT(p,L); \- \\
	 end\- \\
    end {delete} \\
\end{tabbing}
\end{algorithm}

By removing the element from the list, the list is modified and therefore the indexing of the list changes. This is not accounted for when incrementing through all of the positions
for example:
\begin{lstlisting}

list =[12,32,13,13,46]
x = 13

when p = 2:
	DELETE(p,L)		\# list = [12,32,13,46]	list.len = 4
	p = next(p,L)	\# p = 3		list[3] = 46	list[2] = 13

\end{lstlisting}
\ When p is incremented after an element is removed, the following element is skipped. This occurs because the list length is decreased at the same time as the list iterator is being
incremented.
	\\\\\ Revised Version
	\begin{algorithm}
	\KwData{One elementtype and one list as input}
	\KwResult{List with element type removed}
\begin{tabbing}
 proced\=ure delete(elementType x, list l) \+ \\
	  p = position \\
	  whil\=e p != END(L) do begin \+ \\
		if RE\=TRIEVE(p,L) = x then \+\\
			DELETE(p,L); \-\\
		else then\+ \\  
		p = NEXT(p,L); \-\- \\
	 end\- \\
    end {delete} \\
\end{tabbing}
\end{algorithm}
This fix only increments the list iterator if an element is not removed, this ensures no elements are skipped.
\\\\\\\\\\


		\section{Problem 2.11}
		
		\ Times each function is hit\\
		\begin{enumerate}
			\item[FIRST:] 
			\begin{equation*}= \frac{n(n+1)}{2}	\end{equation*}
			\item[NEXT:] \begin{equation*} n+\frac{n(n+1)}{2}+\sum_{1}^{n}n^2\end{equation*}
						\item[LAST:] \begin{equation*} \end{equation*}
						\begin{equation*} n+1	\end{equation*}
		\end{enumerate}



	\end{document}
\end{article}